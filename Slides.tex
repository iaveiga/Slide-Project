\documentclass[9pt]{beamer}
\usetheme{CambridgeUS}%establece el tipo de diapositiva
\usepackage[activeacute,spanish]{babel}
\usepackage{graphicx}
\usepackage[utf8]{inputenc}

\title{NotifyMe}
\author{Iván Aveiga \\ Wilson Enriquez \\ Andrés Sornoza}
\institute{Escuela Superior Politécnica del Litoral}


\begin{document}
	\begin{frame}
		\begin{center}
			\includegraphics[width=0.20\textwidth]{android.png}
		\end{center}
		\titlepage
		\scriptsize
	\end{frame}	
	
	\begin{frame}
		\frametitle{NotifyMe}
			\begin{block}{\textbf{El Problema}}
				La falta de una aplicación en el mercado que permita realizar recordatorios de actividades a realizar dada una lista de actividades
				en diversos lugares geográficos. Ejemplo:
					\begin{itemize}
						\item Recordar la compra de útiles escolares cuando se está cerca de una papelería.
						\item Pago de servicios básicos cuando se está cerca de un lugar de recaudación.
						\item Visitas planificadas a personas o lugares.
					\end{itemize}
			\end{block}
	\end{frame}
	
	\begin{frame}
		\frametitle{NotifyMe}
			\begin{block}{\textbf{La Solución}}
				Desarrollar una aplicación móvil que combine el uso de \emph{GPS} con sistemas \emph{To Do}. \\
				La aplicación permite el ingreso de tareas a realizar con sus respectivas ubicaciones geográficas y recordar
				que tiene que realizar una actividad cuando se está cerca $($ dado un rango previamente $)$ del lugar                               geográfico.
		 \end{block}
	\end{frame}
	
	\begin{frame}
		\frametitle{NotifyMe}
			\begin{block}{\textbf{Especificaciones Técnicas}}
				\begin{itemize}
					\item \emph{\textbf{OS:}} Android
					\item \emph{\textbf{Versión:}} 2.3.6
					\item \emph{\textbf{Lenguaje de Desarrollo:}} Java				
					\item \emph{\textbf{API:}} Google Maps
				\end{itemize}
				\begin{center}
					\includegraphics[width=0.20\textwidth]{gmaps.png}
					\includegraphics[width=0.20\textwidth]{gingerbread.png}
				\end{center}
			\end{block}
	\end{frame}
\end{document}